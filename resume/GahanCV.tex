%%%%%%%%%%%%%%%%%%%%%%%%%%%%%%%%%%%%%%%%%
% Medium Length Professional CV
% LaTeX Template
% Version 2.0 (8/5/13)
%
% This template has been downloaded from:
% http://www.LaTeXTemplates.com
%
% Original author:
% Trey Hunner (http://www.treyhunner.com/)
%
% Important note:
% This template requires the resume.cls file to be in the same directory as the
% .tex file. The resume.cls file provides the resume style used for structuring the
% document.
%
%%%%%%%%%%%%%%%%%%%%%%%%%%%%%%%%%%%%%%%%%

%----------------------------------------------------------------------------------------
%	PACKAGES AND OTHER DOCUMENT CONFIGURATIONS
%----------------------------------------------------------------------------------------

\documentclass{resume} % Use the custom resume.cls style

\usepackage[left=0.4 in,top=0.4in,right=0.4 in,bottom=0.4in]{geometry} % Document margins
\usepackage{fontawesome}
\usepackage{setspace}
\usepackage{hyperref}
\newcommand{\tab}[1]{\hspace{.2667\textwidth}\rlap{#1}} 
\newcommand{\itab}[1]{\hspace{0em}\rlap{#1}}
\name{Gahan Saraiya} % Your name
%\address{123 Pleasant Lane \\ City, State 12345} % Your secondary addess (optional)
\address{ 
  \faPhone \ +91 7359508173 \\
  \faEnvelope \ \href{mailto: gahansaraiya@gmail.com}{gahansaraiya@gmail.com} \\  
  \faLinkedin \ \href{https://www.linkedin.com/in/gahan-saraiya/}{linkedin.com/in/gahan-saraiya} \\ \faGithub \   \href{https://github.com/gahan9/}{github.com/gahan9} 
}  % Your phone number and email

\hypersetup{
    colorlinks=true,
    linkcolor=blue,
    filecolor=magenta,      
    urlcolor=blue,
    pdfpagemode=FullScreen,
    }
\begin{document}

%----------------------------------------------------------------------------------------
%	OBJECTIVE
%----------------------------------------------------------------------------------------

\begin{rSection}{OBJECTIVE}
{With over 6.5+ years of experience as a Senior Software Engineer, currently working at AMD India Pvt Ltd. for managing AI/ML workload execution on HPC device (GPU). I possess a strong command over Python, HPC, UEFI Firmware, and DevOps, complemented by a thorough understanding of their workflows and architectures. My enthusiasm for open-source contribution and knowledge dissemination reflects my commitment to being a philanthropist in the realm of technology. I am eager to channel my expertise into the architectural design of optimized quantum machine learning algorithms tailored for multi-programming environments, as well as to explore the synergies between these cutting-edge technologies and DevOps practices.}

\end{rSection}




%----------------------------------------------------------------------------------------
%	TECHNICAL STRENGTHS SECTION
%----------------------------------------------------------------------------------------

\begin{rSection}{SOFTWARE TOOLS}

\begin{tabular}{ @{} >{\bfseries}l @{\hspace{6ex}} l }

Languages & Python, C, C++, Shell Scripting, SQL, LaTeX, YAML 
\\
Frameworks & Tensorflow, Pytorch, TensorRT, Pytest, Edk2, PyCUDA, Pandas, 
\\ & UEFI, Selenium 
\\
Architecture Design & Parallel Programming System Architecture, ROCm, HIP Programming
\\
Tools and Services & GitHub Actions, Restful APIs, Teamcity, Terraform,
\\ & Docker, Docker Compose, Jenkins, GiT 
\\
Cloud Technologies & GCP, AWS, Azure, Digital Ocean, JFrog Artifactory, TeamCity, 
\\ & JIRA, AWS Sage Maker
\\
System Level IP & RAS, Compute Paritioning, Boot time optimization

\end{tabular}

\end{rSection}


%----------------------------------------------------------------------------------------

%----------------------------------------------------------------------------------------
%	Experience SECTION
%----------------------------------------------------------------------------------------

\begin{rSection}{Experience}

{\bf Senior System Software Designer at AMD India Pvt Ltd} \hfill {Mar 2024 - Present }
\begin{itemize}
    \item Conducting functional and performance benchmarking of HPC devices (GPUs) by developing and executing comprehensive test suites aligned with industry standards.
    \item Develop optimal GPU performance criteria based on thermal limits, power consumption, memory usage, and core utilization.
    \item Incorporating state-of-the-art industry standards design models, implementing and utilizing frameworks like TensorFlow, PyTorch, and various matrix operations.
    \item Provided precise steps for RAS (Reliability, Availability, and Serviceability) injection on MI2XX and MI3XX platforms.
    \item Designed precise steps for range of firmware capabilities, including partitioning, System Management Interface (SMI), HIP profiling, and parallel execution.
    \item Collaborating with a diverse team of engineers and researchers.
    
\end{itemize}

{\textbf{Senior Firmware Development Engineer at Intel Technology India Pvt Ltd}}  \hfill {Oct 2023 - Mar 2024}
\begin{itemize}
    \item Platform provisioning and infrastructure setup for mass automation of HPC products.
    \item Platform Orchestration Layer design of Universal Scalable Firmware (USF).
    \item Harnessing Azure OpenAI Generative AI adaptation for Firmware Development Productivity for assessment code review, guidance, basic security checks of code.
    % \item Terraform deployments for environment for on-the-go dev environment over kubernetes cluster. (including setting up a node platform from scratch)
    \item GitHub Action and JIRA integration for managing Continuous deployment and test through CloudBees Jenkins with baremetal and docker nodes.
    \item Enhancing firmware-based boot-time performance for notebook devices.
\end{itemize}
 
{\textbf{Firmware Development Engineer at Intel Technology India Pvt Ltd}}  \hfill {Jun 2020 - Sept 2023}
\begin{itemize}
    \item Lead the next generation client platform and architecture design at USF POL alongside the domain ownership for Automation Framework, Platform Provisioning, Applied Generative AI \& Prompt Engineering and Quantum Firmware Design.
    \item The end-to-end Software Development Life Cycle is managed, which includes licensing, legals, and IP planning, as well as applicable Code Security and Static Code Analysis.
    \item Jenkins configuration for Security tools - Snyk, Coverity
    \item Improving github code review hooks for python packaging and linting through bandit and flask as Jeninks pipeline
    \item automated testing of firmware functional validation of Firmware with TeamCity.
    \item Mentoring and knowledge development engagement activities.
\end{itemize}

{\textbf{Firmware Engineer Intern at Intel Technology India Pvt Ltd}}  \hfill {Jun 2019 - May 2020}
\begin{itemize}
    \item BIOS Firmware Development for Client products with the framework EDK2, UEFI Firmware, Involvement with assessing new debugging solutions and OpenBoard Package
\end{itemize}

% {\textbf{Teaching Assistantship at Nirma University}}  \hfill {Aug 2018 - May 2019}
% \begin{itemize}
%     \item Data Mining and Visulaization
% \end{itemize}

% {\textbf{Software Developer at Quixom Technology}}   \hfill {Jul 2017 - Apr 2018}
% \begin{itemize}
%     \item SaaS-Employee management system (chat, surveys, benchmark, analytics...) and Restful API Development; socket programming in python to receive continuous data from IoT device and display chunks of data meaningfully, web portal for rendering dynamic menu content in Kodi
% \end{itemize}

% {\textbf{Freelance Developer}}   \hfill {2015 - Apr 2019}
% \begin{itemize}
%     \item Cloud solution for Data gathering and webscraping with selenium and GCP.
%     \item Designing template and banners for High resolution prints
%     \item Inventory management system for Book store with Django.
% \end{itemize}

\end{rSection}


%----------------------------------------------------------------------------------------
%     PROJECT SECTION
%----------------------------------------------------------------------------------------
\begin{rSection}{Projects}
{\bf Prompt Engineering and Generative AI for firmware stack} \hfill {2023 - 2024}\\
\textbf{Description}: Solution design for accelerated silicon tape-in with accelerating firmware deliveries

\begin{itemize}
    \item Work task summarization with Azure LLM Open AI services (GPT3.5 and GPT4)
    \item Code Review against standard static code analysis and common vulnerabilities finding
    \item Generation of Code Enhancement suggestion 
\end{itemize}

\textbf{Platform Provisioning with Pre-Production Reference Board} \hfill {2023 - 2024}\\
\textbf{Description}: Platform Provisioning with automating most common features for pre-production server processor such as:

\begin{itemize}
    \item Remote BMC and BIOS Firmware Flashing
    \item OS Deployment from artifactory
    \item Boot to EFI Shell and OS
\end{itemize}

\textbf{Universal Scalable Firmware} \hfill {2022 - 2024} \\
\textbf{Description}: Architecture design of Platform orchestration layer for Firmware Configuration

\begin{itemize}
    \item Motivation of workgroup - \href{https://uefi.org/sites/default/files/resources/Firmware%20Configuration%20%E2%80%93%20Past%2C%20Present%2C%20and%20Future_Zimmer.pdf}{Firmware Configuration – Past, Present, and Future}
    \item \href{https://universalscalablefirmware.github.io/documentation/7_yaml_boot_configuration.html}{Firmware Configuration}
\end{itemize}

\textbf{Firmware Automation Framework}  \hfill {2020 - Present} \\
\textbf{Description}: Lead SDL for an automation foundation framework for high scaled volume of silicon devices with edk2 UEFI firmware;

\begin{itemize}
    \item Seamless configuration of system without any manual intervention for BIOS configuration.
    \item SDL activities like legal compliance, security, licensing, static code analysis
    \item Continuous deployment and test with jenkins and tox configurations
    \item Release Artifact management to JFrog Artifactory using GitHub Actions
    \item \href{https://github.com/intel/xml-cli}{https://github.com/intel/xml-cli}
\end{itemize}

\textbf{Multiview Video Summarization} \hfill {2019 - 2020}\\
\textbf{Description}: Summarizing video of multiple view angles of egocentric location in to one to generate effective video summary, minimizing overhead to explore multiple cam feed to discover accidents or anomalies.

% \textbf{Consummating Research Projects using Agile Manifesto} \hfill {2018}\\
% \textbf{Description}: Utilization of Agile Manifesto for consummating research projects. Enlightening the appropriateness of Scrum for the research projects which is a widely used agile manifesto at present in the software industry.

% \textbf{Inventory Management and GST Billing for Mobile Shop} \hfill {2018}\\
% \textbf{Description}: To avoid the headache of manual billing automated GST billing system designed to manage stocks and update according to purchase and sales and generate GST invoice for sales.

\textbf{Streaming Box} \hfill {2017 - 2018}\\
\textbf{Description}: An alternate streaming service to DTH and providing easy customized IPTV streaming service based on Kodi (formerly xbmc) Framework

\begin{itemize}
    \item Backend Streaming Service, authentication and server security with Django Framework and REST API
    \item Web Portal to enable vendor customize Streaming Front-end UI for their users
    \item Cross compilation and Native compilation to enable various platform support as Android, Windows and MacOS
\end{itemize}

\end{rSection}

%----------------------------------------------------------------------------------------


%----------------------------------------------------------------------------------------
%	EDUCATION SECTION
%----------------------------------------------------------------------------------------

\begin{rSection}{Education}

{\bf M.Tech Computer Science} \hfill {Jul 2018 - Jun 2020 }
\\ 
Nirma University, GPA: 8.49

{\bf B.E. in Computer Science and Engineering} \hfill {Aug 2013 - Apr 2017 }
Gujarat Technological University, CGPA: 8.08

{\textbf{12th Science, GSHSEB, Class XII}}  \hfill {Jun 2011 - Mar 2013 }\\
Science Stream, 74.92\% 
 
{\textbf{GSHSEB, Class X}}  \hfill {Jun 2010 - Mar 2011 }\\
81.4\%

%Minor in Linguistics \smallskip \\
%Member of Eta Kappa Nu \\
%Member of Upsilon Pi Epsilon \\

\end{rSection}

%----------------------------------------------------------------------------------------
%     Achievements
%----------------------------------------------------------------------------------------
\begin{rSection}{Achievements}

\begin{tabular}{ @{} >{\bfseries}l @{\hspace{6ex}} l }
2023 & Firmware Configuration Presentation at UEFI Developers Conference \& Plugfest \\
2023 & Tech Talk on Harnessing Multiprogramming across CPU, GPU and Quantum Processing Unit \\
2019-Present & 4000+ reputation on Stackoverflow \\
2017-Present & Active volunteer on Peer Learning, Cultural and team building activity \\
2018 & Rank \#1 for Python challenges on hackerrank \\
2018 & Rank \#112 from 2lacs+ participants in Techgig's National Coding Contest 2018 \\
% 2014 & Establishment of complete network for Department

\end{tabular}

\end{rSection}
%----------------------------------------------------------------------------------------

%----------------------------------------------------------------------------------------
%     Publications
%----------------------------------------------------------------------------------------
\begin{rSection}{Publications}

\begin{itemize}
    % \item A Review on Honeypot : Track the Attack. NCCICT, 2016 \hfill {2016}
    \item \href{https://uefi.org/sites/default/files/resources/Firmware%20Configuration%20%E2%80%93%20Past%2C%20Present%2C%20and%20Future_Zimmer.pdf}{Firmware Configuration – Past, Present, and Future}. Presentation, UEFI DevCon 2023 \hfill {2023}
\end{itemize}


\end{rSection}


\end{document}

